\chapter{Question 2}
\section{Questions}
The function $\arcsinh$ is the inverse function of $\sinh$.

\begin{enumerate}
  \item Find the derivative of the function given by
  $x \mapsto \sinh = \frac{e^x - e^{-x}}{2}$. Show that the derivative is
  everywhere (strictly) positive. Infer that $\sinh$ is injective. What is the
  range of this function?
  \item Use the definition of $\cosh x$ and $\sinh x$ to show that
  $\cosh^2 x - \sinh^2 x = 1$.
  \item Use the chain rule to find the derivative of the function determined by
  $x \mapsto \arcsinh x$.
\end{enumerate}

\section{Solutions}
\begin{enumerate}
\item From question 1, part (f):
\begin{align}
  \sinh x &= \frac{1}{2}e^x - e^{-x} \\
  \left(\sinh x\right)' &= \frac{1}{2} e^x + e^{-x} \\
  &= \cosh x
\end{align}

Also from Q1, part (b) section(iii) and part (c): \\
Given that $\left(\sinh x\right)' = \cosh x = \frac{1}{2} e^x + e^{-x}$ and
that both $e^x$ and $e^{-x}$ are strictly positive, they will sum to a strictly
positive number. Halving a positive number does not change its sign. Therefore
$(\sinh x)'$ is strictly positive.
The minimum of $\cosh x$ was shown to be 1 in Question 1, therefore, the
derivitive is \emph{always} equal to or greater than 1.

$\sinh x$ is injective because it is one-to-one and can be mapped back from a
value in its range back to the original $x$ input by way of its inverse
function, $\arcsinh x = \ln(x + \sqrt{1 +x^2})$.


"What is the range of this function?" Unsure if asking for range of $\cosh x$
or $\sinh x$, so have both: \\
Range($\cosh x$): $x \in \mathbb{R} | x \geq 1$. \\
Range($\sinh x$): $x \in \mathbb{R}$. \\

\item Show $\cosh^2x - \sinh^2x =1$
\begin{align}
  \cosh^2x &= (\cosh x)(\cosh x) \\
  \sinh^2x &= (\sinh x)(\sinh x)
  \intertext{So}
  1 &\stackrel{?}{=} \cosh^2x - \sinh^2x \\
    &\stackrel{?}{=} (\cosh x)(\cosh x) - (\sinh x)(\sinh x) \\
    &\stackrel{?}{=}
      \left(\frac{e^x + e^{-x}}{2}\right)^{2}
      - \left(\frac{e^x + e^{-x}}{2}\right)^{2} \\
    &\stackrel{?}{=}
      \left(\frac{e^{2x} +2 + e^{-2x}}{4}\right)
      - \left(\frac{e^{2x} -2 + e^{-2x}}{4}\right) \\
  4 &\stackrel{?}{=}
      \left(e^{2x} +2 + e^{-2x}\right)
      - \left(e^{2x} -2 + e^{-2x}\right) \\
    &\stackrel{?}{=}
      2 + 2 \\
  4 &= 4
\end{align}
LHS = RHS \qedbitches

\item Seek to differentiate $\arcsinh x$ using chain rule.
\begin{align}
  \left(\arcsinh x\right)' &= \left( \ln\left(x + \sqrt{1+x^2}\right) \right)'
  \intertext{Let}
  f &= \ln(x) \\
  g &= x + \sqrt{1+x^2} \\
  \intertext{Such that}
  f' &= \frac{1}{x} \\
  g' &= 1 + \left(\sqrt{1+x^2}\right)' \quad \text{Suddenly, a nested chain rule appears!} \label{eq:q2_nested}
  \intertext{Let}
  h(x) &= \sqrt{x} \\
  j(x) &= 1+x^2
  \intertext{Such that}
  \left(\sqrt{1+x^2}\right)' &= ( h(j(x)) )' = h'(j(x)) \cdot j'(x) \\
  h'(x) &= \frac{1}{2\sqrt{x}} \cdot j'(x) \\
  j'(x) &= 2x
  \intertext{So}
  h'(j(x)) \cdot j'(x) &= \frac{1}{2\sqrt{1+x^2}} \cdot 2x \\
  h'(j(x)) \cdot j'(x) &= \frac{x}{\sqrt{1+x^2}}
  \intertext{Apply back in \eqref{eq:q2_nested}}
  g' &= 1 + \frac{x}{\sqrt{1+x^2}}
  \intertext{Apply this to original problem:}
  \left( \ln\left(x + \sqrt{1+x^2}\right) \right)'
    &= f'(g(x)) \cdot g'(x) \\
    &= \frac{1}{x + \sqrt{1+x^2}} \cdot \left(1 + \frac{x}{\sqrt{1+x^2}}\right) \\
  \intertext{TODO: some magic happens here... I know it happens because
  Wolfram|Alpha confirmed the above line to equal the next line... which is the
  final line!}
    &= \frac{\left(1+\frac{x}{\sqrt{1+x^2}}\right)}{x+\sqrt{1+x^2}} \\
    &= \frac{1}{\sqrt{x^2+1}}
\end{align}\qedbitches
\end{enumerate}
